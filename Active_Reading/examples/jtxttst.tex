\documentclass{article}
\usepackage{amsmath}
\usepackage
    [tight,designi]{web}
\usepackage{exerquiz}
\usepackage[indefIntegral]{dljslib}

\title{Extending the Exerquiz Package Text Fill-in Questions}
\author{D. P. Story}
\subject{Sample file}
\keywords{LaTeX, PDF, derivative, calculus, JavaScript}

\university{NORTHWEST FLORIDA STATE COLLEGE\\
Department of Mathematics}
\email{dpstory@uakron.edu}
\version{2.0}
\copyrightyears{1999-\the\year}

\nocopyright
\norevisionLabel

\def\D{\dfrac {\hbox{\text{d}}}{\text{d}x}}
\def\dPose#1{$\D#1=$ }

\newcommand\redpoint{\par\ifdim\lastskip>0pt\relax\vskip-\lastskip\fi
\vskip\medskipamount\noindent
  \makebox[\parindent][l]{\large\color{red}$\blacktriangleright$}}
\newcommand{\cs}[1]{\texttt{\char`\\#1}}

\useMCCircles

\begin{document}

\maketitle

\section{Text Fill-in Questions}

\textsf{Exerquiz} can now create text fill-in questions, questions
in which the response is text (as opposed to a mathematical
expression). The underlying JavaScript compares the user's
response against acceptable alternatives, as supplied by the
author of the question. If there is a match, the response is
deemed correct. For example:

\redpoint\begin{oQuestion}{ex1}%
Who was the first President\footnote{of the United States}?\
\RespBoxTxt{0}{0}{4}{George Washington}{Washington}
    {G. Washington}{Geo. Washington}
\end{oQuestion}

\medskip\noindent
The command \cs{RespBoxTxt} is the one that creates a text fill-in question. Its
syntax is
\begin{verbatim}
\RespBoxTxt[#1]#2#3[#4]#5<plus listing of alternatives>
\end{verbatim}
\noindent\textbf{\hypertarget{paraRespBoxTxt}{Parameters:}}
\begin{enumerate}
\item[\ttfamily\#1:]Optional parameter used to modify the appearance of the
    text field.
\item[\ttfamily\#2:]This required parameter is a number that indicates
the filtering method to be used. Permissible values of this parameter are
\begin{enumerate}
    \item[\ttfamily-1:] (The default) The author's and user's answers are not filtered
    in any way. (Spaces, case, and punctuation are preserved.)
    \item[\ttfamily0:] The author's and user's answers are converted to
    lower case, any white space and non-word characters are removed.
    \item[\ttfamily1:] The author's and user's answers are converted to
    lower case, any white space is removed.
    \item[\ttfamily2:] The author's and user's answers are stripped of
    white space.
\end{enumerate}
See the JavaScript function \texttt{eqFilter} in \texttt{exerquiz.dtx} for
program code details.  Additional filtering options may be added.

\item[\ttfamily\#3:] This parameter is a number that indicates the compare
method to be used. Permissible values of this parameter are
\begin{enumerate}
    \item[\ttfamily0:] (The default) The author's and user's answers are
    compared for an exact match. (These answers are filtered before they are
    compared.)
    \item[\ttfamily1:] The user's response is searched in an attempt to
    get a substring match with the author's alternatives. Additional comparison
    methods may be added.
\end{enumerate}
See the JavaScript function \texttt{compareTxt} in
\texttt{exerquiz.dtx} for the program code details.

\item[\ttfamily\#4:] Optional, a named destination to the
solution to the question. If this parameter appears, then a
solution must follow the question, enclosed in a \texttt{solution}
environment.  If the forth parameter is a `\texttt*', then an automatic
naming scheme is used instead.
\item[\ttfamily\#5:] This required parameter is the
number of alternative answers that are acceptable. The alternative
answers are listed immediately after this parameter. (The example
above specified that $4$ alternatives follow.)
\end{enumerate}

\makeatletter
% remove label `Quiz' and gobble up the space that follows it.
\renewcommand\sqlabel[1]{\@gobble}
\makeatother
\begin{shortquiz}[comboexamp]%
The following series of examples illustrate different combinations
parameters~\texttt{\#2} and~\texttt{\#3}.  All questions are in
response to the question ``Who was the first president of the
United States?''
\begin{questions}
\item Remove all white space and non-word characters, convert to lower case,
then look for a match.
\RespBoxTxt{0}{0}{4}{George Washington}{Washington}
    {G. Washington}{Geo. Washington}
Of course, ``George Washington'' and ``G. Washington'' are correct, but
so too are ``georgewashington''and ``gwashington''.

\item Remove all white space, convert to lower case, then look for an
exact match. Here we don't remove non-word characters, such as punctuation.
\RespBoxTxt{1}{0}{4}{George Washington}{Washington}
    {G. Washington}{Geo. Washington}
For example, answers ``G. Washington'', ``georgewashington'' and
``g. washington'' are correct, but ``gwashington'' is not.

\item Remove all white space, then look for an exact match. (Here,
we do not remove punctuation and do not convert to lower case.)
\RespBoxTxt{2}{0}{4}{George Washington}{Washington}
    {G. Washington}{Geo. Washington} The
response ``G. Washington'' is correct, but ``g. washington'' is not.

\item Now lets put parameter~\texttt{\#3} equal to \texttt1. Here, we convert
to lower case, remove white space, and look for a substring match.
\RespBoxTxt{1}{1}{4}{George Washington}{Washington}
    {G. Washington}{Geo. Washington} Note that ``President Washington'',
``General Geo. Washington'', and ``Washington, George'' are correct.
Also ``Fred Washington'' is correct, since it matches the second
alternative, Washington. (If we eliminate Washington as an
alternative, then ``Fred Washington'' would be judged incorrect,
let's test that theory:
\begin{oQuestion}{wash2}\RespBoxTxt{2}{1}{3}{George Washington}
{G. Washington}{Geo. Washington}\end{oQuestion}
\end{questions}
\end{shortquiz}

\section{Short Quiz Environment}

\begin{shortquiz}
Answer each of the following. Passing is 100\%.

\begin{questions}

\answersEndHook{\hfill\makebox[0pt][r]{\sqTallyBox}}

\item Who was the first president of the United States?\par\kern3pt\noindent
\RespBoxTxt{0}{0}[geow]{4}{George Washington}{Washington}{G. Washington}{Geo. Washington}\hfill
\CorrAnsButton{George Washington}\kern1bp\sqTallyBox
\begin{solution}
Yes, George Washington was the first President of the United
States of America.
\end{solution}

\item Name \emph{one} of the two people recognized as a founder of
Calculus.\par\kern3pt\noindent
\RespBoxTxt{2}{0}{5}{Isaac Newton}{Newton}{I. Newton}{Gottfried Leibniz}{Leibniz}\hfill
\CorrAnsButton{Isaac Newton or Gottfried Leibniz}\kern1bp\sqTallyBox

\item If $f$ is differentiable, then $f$ is continuous.
\begin{answers}{4}
\Ans1 True &
\Ans0 False
\end{answers}

\item
\dPose {4 x^{-1/2}}\RespBoxMath{-2*pow(x,-3/2)}{4}{.0001}{[1,2]}\hfill
\CorrAnsButton{-2*x^(-3/2)}\kern1bp\sqTallyBox

\item
$\displaystyle\int \frac 1x\,dx = $\space
\RespBoxMath{ln(abs(x))}{4}{.0001}{[1,2]}[indefCompare]\hfill
\CorrAnsButton{ln(|x|)}\kern1bp\sqTallyBox

\end{questions}
\end{shortquiz}
\begin{flushright}
\sqClearButton\kern1bp\sqTallyTotal
\end{flushright}

\newpage
\section{Quiz Environment}

Here is a mixture of all types of questions, all with solutions.

\useBeginQuizButton[\CA{Begin}]
\useEndQuizButton[\CA{Finish}]

\begin{quiz}*{calcquiz} Answer each of the following. Passing
is 100\%.

\begin{questions}

\item If $\lim_{x\to a} f(x) = f(a)$, then we say that $f$ is\dots
\begin{answers}[cont]3
\Ans0 differentiable &\Ans1 continuous &\Ans0 integrable
\end{answers}
\begin{solution}
A function $f$ is said to be continuous at $x=a$ if $x\in\operatorname{Dom}(f)$,
$\lim_{x\to a} f(x) $ exists and $\lim_{x\to a} f(x) = f(a)$.
\end{solution}

\item Name \emph{one} of the two people recognized as a founder of
Calculus.\par\kern3pt
\RespBoxTxt{0}{0}[calc]{4}{Isaac Newton}{Newton}{Gottfried Leibniz}{Leibniz}%
\CorrAnsButton{Isaac Newton or Gottfried Leibniz}
\begin{solution}
Isaac Newton and Gottfried Leibniz are the co-creators of Calculus.
\end{solution}

\item $\cos(\pi) = \RespBoxMath{-1}[cospi]{1}{.0001}{[2,4]}\CorrAnsButton{-1}$
\begin{solution}
Oh, come on now. You know that $\cos(\pi)=-1$.
\end{solution}

\item $\displaystyle\int \sin(x)\,dx =
\RespBoxMath{-cos(x)}[intSin]{4}{.0001}{[0,1]}[indefCompare]\CorrAnsButton{-cos(x)}$
\begin{solution}
\relax\begin{equation*}
        \int \sin(x) \,dx = -\cos(x) + C
\end{equation*}
\adjDisplayBelow
\end{solution}

\end{questions}
\end{quiz}\quad\ScoreField\currQuiz\eqButton\currQuiz

\noindent
Answers: \AnswerField\currQuiz

\end{document}
