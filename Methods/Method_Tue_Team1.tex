\documentclass{article}\usepackage[]{graphicx}\usepackage[]{color}
%% maxwidth is the original width if it is less than linewidth
%% otherwise use linewidth (to make sure the graphics do not exceed the margin)
\makeatletter
\def\maxwidth{ %
  \ifdim\Gin@nat@width>\linewidth
    \linewidth
  \else
    \Gin@nat@width
  \fi
}
\makeatother

\definecolor{fgcolor}{rgb}{0.345, 0.345, 0.345}
\newcommand{\hlnum}[1]{\textcolor[rgb]{0.686,0.059,0.569}{#1}}%
\newcommand{\hlstr}[1]{\textcolor[rgb]{0.192,0.494,0.8}{#1}}%
\newcommand{\hlcom}[1]{\textcolor[rgb]{0.678,0.584,0.686}{\textit{#1}}}%
\newcommand{\hlopt}[1]{\textcolor[rgb]{0,0,0}{#1}}%
\newcommand{\hlstd}[1]{\textcolor[rgb]{0.345,0.345,0.345}{#1}}%
\newcommand{\hlkwa}[1]{\textcolor[rgb]{0.161,0.373,0.58}{\textbf{#1}}}%
\newcommand{\hlkwb}[1]{\textcolor[rgb]{0.69,0.353,0.396}{#1}}%
\newcommand{\hlkwc}[1]{\textcolor[rgb]{0.333,0.667,0.333}{#1}}%
\newcommand{\hlkwd}[1]{\textcolor[rgb]{0.737,0.353,0.396}{\textbf{#1}}}%

\usepackage{framed}
\makeatletter
\newenvironment{kframe}{%
 \def\at@end@of@kframe{}%
 \ifinner\ifhmode%
  \def\at@end@of@kframe{\end{minipage}}%
  \begin{minipage}{\columnwidth}%
 \fi\fi%
 \def\FrameCommand##1{\hskip\@totalleftmargin \hskip-\fboxsep
 \colorbox{shadecolor}{##1}\hskip-\fboxsep
     % There is no \\@totalrightmargin, so:
     \hskip-\linewidth \hskip-\@totalleftmargin \hskip\columnwidth}%
 \MakeFramed {\advance\hsize-\width
   \@totalleftmargin\z@ \linewidth\hsize
   \@setminipage}}%
 {\par\unskip\endMakeFramed%
 \at@end@of@kframe}
\makeatother

\definecolor{shadecolor}{rgb}{.97, .97, .97}
\definecolor{messagecolor}{rgb}{0, 0, 0}
\definecolor{warningcolor}{rgb}{1, 0, 1}
\definecolor{errorcolor}{rgb}{1, 0, 0}
\newenvironment{knitrout}{}{} % an empty environment to be redefined in TeX

\usepackage{alltt}

\title{Field methods for determining distribution of red algae in Santa Ana River}
\IfFileExists{upquote.sty}{\usepackage{upquote}}{}
\begin{document}
\maketitle

\section{Materials and Equipment}

Analogue pyrex thermometer

GPS 

Swim goggles 

Densiometer 

5 containers for algae specimens, from EA laborary 

30cm x 30cm Quadrat

30 m measuring tape

Random number generator: https://www.random.org/

\section{Procedures}

Site selection: 3 nonrandom reaches (Reaches A-C) were selected on the Santa Ana River due to their distinct physical characteristics. Reach A was farthest down stream, selected at the recommendation of US Fish and Wildlife for its abundance of SA suckers, at N34*2'5" W117*21'17". Reach B was just below the SA River's confluence with the RIX facility's outflow pipe, at N34*2'21" W117*21'20". Reach C was farthest upstream, above the confluence, at N34*2'29" W117*21'15". 

Along each reach, 3 sites 1-10m apart were selected (Sites 1-3) with site 1 being farthest downstream in a reach, site 3 being farthest upstream. A random number generator was used ahead of time to select the distances between sites within a reach. Each site was composed of individual measurements along the right, middle, and lefthand sides of the stream (while facing upstream). 

Each measurement (27 total: right, middle, and left for 3 sites for 3 reaches; 3x3x3=27 measurements) consisted of the following parameters: 

Percent algae: a 30cm x 30cm quadrat was placed into the stream, which was then viewed using goggles. Percent cover was visually estimated 10 percent increments.

Stream bed composition: the relative coarseness of the streambed was categorically characterized as either coarse (gravel and cobbles) or fine (sand and silt).
   
Canopy cover: directly above each alage measurement, the densiometer was used quantify canopy cover. Measurements were recorded as the proportion of dark dots on the densiometer. 

Temperature: the analogue pyrex themometer was submerged into the stream for approximately 1 minute, then read. 

\end{document}
