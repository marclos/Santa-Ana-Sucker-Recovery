\documentclass{article}\usepackage[]{graphicx}\usepackage[]{color}
%% maxwidth is the original width if it is less than linewidth
%% otherwise use linewidth (to make sure the graphics do not exceed the margin)
\makeatletter
\def\maxwidth{ %
  \ifdim\Gin@nat@width>\linewidth
    \linewidth
  \else
    \Gin@nat@width
  \fi
}
\makeatother

\definecolor{fgcolor}{rgb}{0.345, 0.345, 0.345}
\newcommand{\hlnum}[1]{\textcolor[rgb]{0.686,0.059,0.569}{#1}}%
\newcommand{\hlstr}[1]{\textcolor[rgb]{0.192,0.494,0.8}{#1}}%
\newcommand{\hlcom}[1]{\textcolor[rgb]{0.678,0.584,0.686}{\textit{#1}}}%
\newcommand{\hlopt}[1]{\textcolor[rgb]{0,0,0}{#1}}%
\newcommand{\hlstd}[1]{\textcolor[rgb]{0.345,0.345,0.345}{#1}}%
\newcommand{\hlkwa}[1]{\textcolor[rgb]{0.161,0.373,0.58}{\textbf{#1}}}%
\newcommand{\hlkwb}[1]{\textcolor[rgb]{0.69,0.353,0.396}{#1}}%
\newcommand{\hlkwc}[1]{\textcolor[rgb]{0.333,0.667,0.333}{#1}}%
\newcommand{\hlkwd}[1]{\textcolor[rgb]{0.737,0.353,0.396}{\textbf{#1}}}%

\usepackage{framed}
\makeatletter
\newenvironment{kframe}{%
 \def\at@end@of@kframe{}%
 \ifinner\ifhmode%
  \def\at@end@of@kframe{\end{minipage}}%
  \begin{minipage}{\columnwidth}%
 \fi\fi%
 \def\FrameCommand##1{\hskip\@totalleftmargin \hskip-\fboxsep
 \colorbox{shadecolor}{##1}\hskip-\fboxsep
     % There is no \\@totalrightmargin, so:
     \hskip-\linewidth \hskip-\@totalleftmargin \hskip\columnwidth}%
 \MakeFramed {\advance\hsize-\width
   \@totalleftmargin\z@ \linewidth\hsize
   \@setminipage}}%
 {\par\unskip\endMakeFramed%
 \at@end@of@kframe}
\makeatother

\definecolor{shadecolor}{rgb}{.97, .97, .97}
\definecolor{messagecolor}{rgb}{0, 0, 0}
\definecolor{warningcolor}{rgb}{1, 0, 1}
\definecolor{errorcolor}{rgb}{1, 0, 0}
\newenvironment{knitrout}{}{} % an empty environment to be redefined in TeX

\usepackage{alltt}

\title{Videography and the Santa Ana Sucker}
\author{Clare Flynn and Wendy Norena}
\IfFileExists{upquote.sty}{\usepackage{upquote}}{}
\begin{document}
\SweaveOpts{concordance=TRUE}

\maketitle

\newpage
\tableofcontents
\newpage

\section{Introduction}


\subsection{Problem Statement}


\subsection{Background (Literature Review)}
The Santa Ana sucker is a 16 cm fish that lives in the rivers of Southern California.  They have recently been placed on the endangered species list, partially due to their losing over 70 percent of their habitat (FWS 2012). Also, in the 1960s, the habitat of the suckers would range from 10-26°C, but when we tested the water in September 2016, we found temperatures in the Santa Ana river up to 35.5°C (Greenfield & Ross & Deckert 1970).  This rise in temperature is most likely due to the extreme industrialization of the stream; much of it runs over concrete, which heats up to extreme temperatures in the sunlight.  The river is also greatly diminished from what it used to be due to the extreme drought in Southern California.  The river is shallower, slower moving, and has less ice melt coming from the mountains, all of which factor in to an increase in temperature.  Our goal is to discover whether or not the Santa Ana suckers are coping with this dramatic increase in temperature by moving to cooler sections of the stream throughout the day.  We expect that water temperatures at the downstream location will be less than the upstream location throughout the day. We know the current range of the Santa Ana Sucker differs siginificantly from its historic distribution in the Los Angeles Basin (Thompson 2010). We also know that the sucker prefers environments with cool water (Moyle 2002) and other fish have been known to regulate their body temperatures by moving to different areas in their habitat throughout the day (citation needed).
Greenfield, D. W., Ross, S. T., & Deckert, G. D. (1970). Some aspects of the life history of the Santa Ana sucker, Catostomus (Pantosteus) santaanae (Snyder). Calif. Fish Game, 56(3), 166-179.

\subsection{Background Research w/Citations} 



\subsection{Objectives}

\subsection{Materials and Equipment}
\begin{itemize}
\item 2 Waterproof GoPro Hero 4 Silver cameras with mounts
\item 4 64GB microSD SanDisk memory cards
\item 4 Waterproof Re-Fuel 6-Hour ActionPack Battery for GoPro HERO by DigiPower
\item 2 HOBO Tidbit water temperature data loggers
\item 2 Grey Cinder block cubes open on two parallel sides, ~8in x 8in x 8in, Home Depot
\item 2 Grey Cinder block backs, ~8in x 8in
\item 1 bottle of Original Sticks to Everything Gorilla Glue
\end{itemize}

\subsection{Methods}
\paragraph{Preparing the Equipment}
First, we acquired all the necessary equipment, keeping in mind the length of time we wanted to keep our cameras underwater. We chose the GoPro Hero 4 Silver because of its battery life and recording time. To set up the cameras, we removed them from the packaging, inserted a microSD card into each, and charged them fully by connecting the included USB cables to a computer. The cameras needed to be fully charged before we were able to adjust the settings. Once the cameras were sufficiently charged, we set the filming settings to record in 720p x 30fps. We set the cameras aside and left them charging. 

Next, we charged all four of the waterproof Re-Fuel battery packs. As these were charging, we put together our cinder block mounting structures. We took our cinder block cubes and set them up on a clean, stable table. We connected the flat adhesive mounts that were included with the GoPros and connected a GoPro camera to each. Since the cinder blocks were open on two parallel sides, we were able to see right through the cinder block cube to the other side. One of us stood on one side with the GoPro and mount, and the other stood on the other side of the opening. One of us turned the GoPro on and put the camera with the mount inside the cinder block cube, using the view on the screen to find the best position for the mount inside of the cube, taking care to ensure we could clearly see the other person on the other side. We found an ideal place where the view was mostly unobstructed by the sides of the cube but the cameras were still far enough inside the cube that they wouldn’t be too easily spotted by passersby. This spot was 5 cm from the edge of the cube. We traced the front and the back of the mount so we could glue it in the correct place. We repeated this procedure for the second cube and mount, and standardized the construction by placing the mount in the second cube 5 cm away from the edge of the cube. 

In order to securely glue the mount in place, we used Gorilla glue. To activate the Gorilla glue, we first had to moisten one of the surfaces with water. We moistened the mounts on the adhesive side. We did not remove the adhesive backing so we could reutilize the mounts in the future. Once the mount was damp, we put Gorilla glue on the cinder block inside the lines we had drawn around the mount. We then placed the mount on the Gorilla glue, taking care to align the edges of the mount with the lines we had drawn. Next, as per the Gorilla glue instructions, we found a heavy object that could provide significant pressure on the mount and that would fit inside the cube. We left this for three hours to harden.

Upon returning to the lab, we removed the heavy objects from the cube and checked the seal on the mount and cinder block to ensure the bond had successfully cemented. After this, we went to work on attaching the cinder block backs to close up one of the open sides on the cubes. We repeated much of the same process we used when attaching the mounts to the cubes, and followed the Gorilla Glue instructions carefully. First, one of us moistened the cinder block back while the other applied Gorilla Glue to the edge of the cube. Then, we carefully aligned the corners of the back with the cube. We repeated this with the second cube. Seeing that the cinderblock back was heavy enough on it’s own, we did not place a heavy object on top of this structure and instead simply left it to dry and harden overnight.
Finally, we checked on the cameras and battery packs again to ensure they had charged. We left them plugged in overnight. We also packed away the Gorilla Glue, the multiple SD cards, paper towels, and extra mounts in a field kit so we could deal with any emergencies in the field. 
\paragraph{Field Work}
We started our first recording session at 10 am.  We drove to the downstream site and found a spot under brush cover in a pool next to a fast moving section of the stream.  We first placed the cinderblock squarely on the riverbed and positioned it facing the fast moving water.  We then turned the camera on, pressed record, and placed it on the mount in the cinderblock.  We let it run for a few seconds, then took it out and watched the video to ensure it was recording at a good angle.  We then pressed record again and replaced it.  Before leaving, we marked the area with flags so we would be able to find it again.

Next, we walked approximately 20 minutes upstream to another covered pool next to a fast moving section, and repeated the camera placement procedures.  We marked with flags, and then left.

We returned at approximately 2 pm.  We took out the camera at the downstream site, replaced the memory card and the battery pack, hit record, and replaced the camera exactly as it was positioned previously.  We then walked upstream and did the same thing with the second camera.  After returning, we cleared the memory cards and plugged in the battery packs.

Our last recording session was at 8 am the next morning.  We replaced the memory cards and battery packs again and returned the cameras to their positions.  One of us returned the next day to collect the cameras.

\subsection{Site Descrition}
 
\subsection{Field Methods}

\subsection{Laboratory Methods}

\subsection{Statistical Methods}

\section{Results}


\begin{figure}
\includegraphics{Figures/Temp}
\caption{Temperature time...}
\label{Temp}
\end{figure}

\section{Discussion}


\section{Conclusion and Recommendations}


\end{document}
