\documentclass{tufte-handout}
\usepackage{hyperref}

\title{Project Proposal}
\author{Sophie Janssen}

\usepackage{Sweave}
\begin{document}
\Sconcordance{concordance:EA30_ProjectPlan_Janssen.tex:EA30_ProjectPlan_Janssen.Rnw:%
1 6 1 1 0 45 1}

%\SweaveOpts{concordance=TRUE}

\maketitle The Effect of Temperature on the Santa Ana Sucker

\section EA 030 Science and the Environment

\begin{description}
  \item[Question to be Addressed] How does temperature affect the Santa Ana sucker populations? The Santa Ana sucker will be small in number where water temperatures exceed 32.8 degrees Celsius, as calculated by the US Fish and Wildlife Service. Conversely, more Santa Ana suckers will be found where water temperatures are around 22 degrees, as this is their ideal living temperature. Fish mortality rates rose exponentially in regions above 32.8 degrees C according to work conducted by the US Fish and Wildlife Service USFWS. In the Rialto Channel at Agua Mansa, water is collected from households and daily use, gathered in storage and holding tanks, processed and treated, and then sent back via concrete canals to the main portion of the river. We would like to see if this process, which tends to increase/maintain the temperature of the water every step along the way, truly harms the Santa Ana sucker. The water volume has also been reduced to small depth, measurable in centimeters, such that the water becomes easily heated by the sun and do to the drought, a significant lack of shade has also increased sun exposure. These factors also affect the water temperature and will be looked at if time permits or if another lab can be found outlining the results of such experiments.
  \item[Approach] To evaluate temperature's effect on the Santa Ana sucker, we will place three HOBO pendant data loggers on three different locations along a portion of the Santa Ana river near the Veolia Water North America Sewage Treatment Plant at the juncture of the Agua Mansa Road and the water from the sewage treatment plant, at the juncture of a lower hot-water stream and the main Santa Ana river, and at a pool that combines several tributaries farther downstream, away from the plant. The loggers will be placed at the same depth and in the shade, to avoid variation due to sun exposure and stratification. We will need to go into the field twice: ideally on September 21 to place the HOBO loggers in the field and on September 28 to collect the loggers. During this week, the loggers will continuously collect data every fifteen minutes, day and night. Only the project workers, Nicole and myself, are needed to place the loggers and to collect them. However, additional workers might be needed to check-up on the loggers, possibly after three/four days after being placed in the river, to ensure they have stayed in place. This project will require three HOBO Pendant Temperature Data Loggers, an Optic USB Base Station, a standard USB cable, and HOBOware Software. These loggers will remain in their designated locations and record air and water temperature between -20°C to 70°C Air 0°C to 50°C Water, to test varying temperatures in the river based off of runoff from the plant. Additionally, data will be required from US Fish and Wildlife Service on past fish population census conducted in this portion of the Santa Ana river, preferably as recent as possible. I acknowledge we have the resources to conduct this work at our disposal.
  \item[Cited Methods] Albertson, L.K., Koenig, L.E., Lewis, B.L., Zeug, S.C., Harrison, L.R., & Cardinale, B.J. (2012). How Does Restored Habitat for Chinook Salmon (Oncorhynchus Tshawytscha) in the Merced River in California Compare with other Chinook Streams? River Research and Applications, 29(4), 469-482). doi: 10.1002/rra.1604
By looking at Chinook salmon in the Merced river, restoration projects seemed to be failing to prevent the Chinook population from falling. The installation of gravel augmentation in a reconfigured channel seemed to have little impact on the salmon, suggesting that other factors were catalyzing the fall of the species. By comparing the restored portion with other portions of the Merced river, food web characteristics and flow discharge seemed to produce the same results on the various life stages of the salmon. However, higher temperatures, less woody debris, and minimal riparian cover seemed to limit populations in the restored portions. Restoration efforts are then presented with an added challenge of ensuring that every aspect of the ecosystem is beneficial to the species, which demands more work toward temperature regulation and attempts to restore the river bank. To see how the Santa Ana sucker would react to similar conservation efforts would be interesting in discussions in attempting to determine solutions. 
Coulter, D. P., Höök, T. O., Mahapatra, C. T., Guffey, S. C., & Sepúlveda, M. S. (2015). Fluctuating Water Temperatures Affect Development, Physiological Responses and Cause Sex Reversal in Fathead Minnows. Environmental Science & Technology, 49(3), 1921-1928. doi:10.1021/es5057159
Human activities can increase water temperature. Water-based organisms are sensitive to temperature change, especially young fish due to limited mobility. This paper explained how young Flathead Minnows exposed to warmer temperatures underwent a nondirectional sex reversal. This paper shows us how temperature can greatly affect fish and stress them out. Clearly, water temperature drastically affects fish, not necessarily in a positive way, and therefore, we should see if there is a correlation between stream temperature and where the Santa Ana Sucker chooses to live.
Los Huertos, Marc. (2016). Thermal Properties of Water. Environmental Science of Aquatic Systems. 297-308.
Temperature varies greatly in its impact on fish depending upon the conditions affecting the lake or river. Heat, temperature, thermal energy, and heat capacity all slightly change how heat is measured in an ecosystem. Water in general has a high heat capacity, which indicates its high specific heat. These aquatic systems therefore often retain their heat and are less susceptible to increase/decrease in temperature. Inflows/mixing can have an effect on water temperature but it is often hard to detect due to thermal stratification mixing, seasonal change in temperature profile depth, and small volume inflow in terms of fraction of the lake volume. This chapter sheds light that variations in temperature trigger chemicals dissolving/remaining, tend to raise/lower a fish’s body temperature to the same degree, etc. Temperature impacts many other features of water quality, which will be important to keep in mind going forward with the project.
Sadler, K. (1980). Effect of the warm water discharge from a power station on fish populations in the river Trent. Journal Of Applied Ecology, 17(2), 349-357.
A power station discharged water that was on average 7 degrees Celsius above normal in the River Trent. This meant that in affected areas, the winter migration was delayed from Sep/Oct to Dec/Jan, while in unaffected areas, migration continued at normal times. Also, below the power station, fish preferred to live further downstream, in terms of diversity and population density. This tells us that fish are able to change their living patterns based on water temperature, and shows that fish were noticeably affected by the temperature change. 
U.S. Fish & Wildlife Service. 2012. Recovery Outline for Santa Ana sucker. Sacramento, California. 38 pp.
This report discusses Santa Ana Suckers and possible recovery plans. In order to do so, it clearly outlines Sucker habitat preferences, behavior, and threats. We used this source to determine preferred temperature for the Santa Ana Sucker and in general to inform ourselves more about the fish. Little research has been done on specific threats to the Santa Ana Sucker, but the document did point out possible threats arising from hydrological modification and urban development in general. Sophie and I thought that potentially water coming out of a treatment plant could be a threat under hydrological modification or urban development, so that will be what our research focuses on.

\end{description}
\section{Background} Increased water temperatures seem to put additional pressure upon the fish and have fatal consequences, from reduced body mass to inability to reproduce (Los Huertos, 2016). Also a reduction in oxygen caused by the invasion of a plant, such as red algae, appears to occur in hotter temperatures, which leads to competition and habitat reduction for the Santa Ana sucker. The Santa Ana sucker strangely depends upon these plants for some water in the streams, as without the plants, a lot of the water may be redirected to agricultural endeavors or toward drinking water. However, the sucker also seems to be failing in these new conditions, probably from a combination of soil quality, water quality, and habitat loss. With this complex relationship between the sucker and the treatment plant in mind, our goal is to determine if the higher water temperatures created and sustained by the plant significantly affect the sucker. If so, we will look for solutions in which to help lower temperatures to a more livable standard for the suckers and be able to concentrate on other variables that could be impacting the Santa Ana sucker. Another point to consider is that of behavioral thermoregulation, which constitutes fish migrating toward areas that are cooler, in the case of Chinook salmon, and that better suit their ectothermic bodies (Albertson, et. al, 2012). Thus, the life cycle of the sucker could be more impacted than individuals, as different stages of life are conducted in different portions of the river, as well as different depths. This concept is important to keep in mind as we may not see a big affect on the health of individuals, who simply moved to cooler pools, but the spawning and development processes of the fish could be significantly altered.

\section{Question to be Addressed} How does temperature impact the Santa Ana sucker population? More specifically, how does the higher water temperature of water treatment runoff impact the habitat environment and quality for the suckers?

\section{Hypothesis/Hypotheses} High temperatures seem to negatively impact fish populations, as a fish's body temperature generally is within a degree or so of the water temperature. Therefore, if the water temperature is above 32.8 degrees Celsius, I predict the number of suckers in this region to be relatively low, while where water temperatures are below 32.8 degrees, closer to 22 degrees (optimal temperature for the sucker), I expect to find more suckers. In terms of geography of the Rialto region, I expect to see less fish where the water first emerges from the plant compared with further downstream where the water collects into a pool. I expect these results to also be impacted by the presence of other chemicals in the water from the sewage plant as well as runoff, some of which may dissolve if temperatures rise. Water temperature cannot be blamed or reasoned as the only cause of sucker population declining.

\section{Approach}
  The purpose of this experiment will be to determine a baseline cycle of the water temperature in a small portion of the Santa Ana river. Ideally, the water temperature would have been collected for many years, and temperatures could be compared, pre-plant to post-plant, to directly see the change, if any, the plant has caused in the water temperature. Assuming a change has occurred, the next step is to determine if this change has been significant enough to affect the sucker population. The US Fish and Wildlife Service has gathered data about the population density of the Santa Ana sucker in various parts of the river. By overlapping this data with our temperature measurements, we can begin to see if there is any correlation between temperature and population. The plant could have caused overall higher water temperatures but not be significant enough to truly cause behavioral thermoregulation or to be fatal. This outcome would be caused by mixing of the layers of the river or simply by the laws of thermodynamics and heat, where water can conduct heat decently well, thereby distributing/dulling the impact of a new substance or raised temperatures. 
 

\section{Methods w/Citations} We selected our portion of the stream because it has been analyzed by the USFWS and therefore, has past data collected, which is not the case with many streams in California. By selecting only three sample sites, we raise our risk of generating results that are atypical or irrelevant, as we may not capture the mean value of any one of our factors for any particular stream (Alberston, et. al). As this experiment runs during the autumn season, seasonal changes, such as hotter air temperatures, may influence the data. The way in which the sucker density was calculated was up to the USFWS which leaves us to assume their methods were consistent and accurate. 

\paragraph{Equipment}
This endeavor will require three HOBO Water Temperature Data Loggers, as well as a BASE U-4 Optic USB Base Station and standard USB cable. In order to access the measurements, the free HOBOware software will be installed on my laptop to read the information. Additionally, transportation will be needed to reach the river to place the loggers and again to retrieve them.

\paragraph{Sampling} Using the parameters established by the USFWS in their data analysis, we will assume the fish population was relatively mixed between females and males, and distributed randomly by age, size, and maturity. The three regions of the river were picked because of their distinct inflows and bottlenecks, which should reveal more information about riffles versus pools in addition to water temperature. The pool furthest downstream will serve as an example of a thriving habitat, based off our prediction that the water farthest from the plant, and therefore cooler, will host higher numbers of fish.


\section{Expected Results and Analysis Methods} I hope to see suckers most abundant in the larger pool, where water temperatures should be closer to the optimal temperature. A negligible correlation between the temperature and Santa Ana sucker population would arise for two main reasons. Primarily, the sucker has declined to such small numbers, only a few hundred in some parts, that there may not be a large enough sample source from which to conclude information. In other words, the sucker population may have dwindled so low in this area that there are very few from which to study. This concern will be addressed once I receive information on population density from the USFWS. Their behavior may be coincidental or random at best, but we are restricted by the dwindling population in determining which is the case. Secondly, many other factors are compounding upon the sucker, some of which are triggered from the rise in temperature. Habitat loss, the introduction of chemicals and other products into the water from the plant, solar radiation, drought, and fragmentation of the river all have dramatically affected the sucker population and can be heightened by an increase in temperature. For example, fragmentation and diversion of the Santa Ana river has lead to a smaller overall flow, disjointed by water storage tanks, concrete canals, and treatment plants. The normal usage of hot water in everyday consumer life and the treatment plant storage tanks raise water temperatures before it evens rejoins the main river. The flow and overall volume of water is sizably less, which allows solar radiation to heat the water faster. Therefore, the source of the heat is split amongst many parties and cannot be traced back to one sole contributor, which makes finding solution a bit more challenging.
  By using the HOBOware, I will be able to compare and contrast the three test sites as well as see how much temperature varies over twenty four hours over one week. Also, I can see these patterns graphically and copy this data into a new project to see how the population numbers compare, via boxplots, etc. More information from the other EA 030 lab groups would supplement this project nicely with information about the red algal blooms and microbial populations, which may ebb and flow with temperature changes as well, and have a significant impact on the water and habitat quality.


\end{document}
