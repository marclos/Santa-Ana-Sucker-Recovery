\documentclass{tufte-handout}\usepackage[]{graphicx}\usepackage[]{color}
%% maxwidth is the original width if it is less than linewidth
%% otherwise use linewidth (to make sure the graphics do not exceed the margin)
\makeatletter
\def\maxwidth{ %
  \ifdim\Gin@nat@width>\linewidth
    \linewidth
  \else
    \Gin@nat@width
  \fi
}
\makeatother

\definecolor{fgcolor}{rgb}{0.345, 0.345, 0.345}
\newcommand{\hlnum}[1]{\textcolor[rgb]{0.686,0.059,0.569}{#1}}%
\newcommand{\hlstr}[1]{\textcolor[rgb]{0.192,0.494,0.8}{#1}}%
\newcommand{\hlcom}[1]{\textcolor[rgb]{0.678,0.584,0.686}{\textit{#1}}}%
\newcommand{\hlopt}[1]{\textcolor[rgb]{0,0,0}{#1}}%
\newcommand{\hlstd}[1]{\textcolor[rgb]{0.345,0.345,0.345}{#1}}%
\newcommand{\hlkwa}[1]{\textcolor[rgb]{0.161,0.373,0.58}{\textbf{#1}}}%
\newcommand{\hlkwb}[1]{\textcolor[rgb]{0.69,0.353,0.396}{#1}}%
\newcommand{\hlkwc}[1]{\textcolor[rgb]{0.333,0.667,0.333}{#1}}%
\newcommand{\hlkwd}[1]{\textcolor[rgb]{0.737,0.353,0.396}{\textbf{#1}}}%

\usepackage{framed}
\makeatletter
\newenvironment{kframe}{%
 \def\at@end@of@kframe{}%
 \ifinner\ifhmode%
  \def\at@end@of@kframe{\end{minipage}}%
  \begin{minipage}{\columnwidth}%
 \fi\fi%
 \def\FrameCommand##1{\hskip\@totalleftmargin \hskip-\fboxsep
 \colorbox{shadecolor}{##1}\hskip-\fboxsep
     % There is no \\@totalrightmargin, so:
     \hskip-\linewidth \hskip-\@totalleftmargin \hskip\columnwidth}%
 \MakeFramed {\advance\hsize-\width
   \@totalleftmargin\z@ \linewidth\hsize
   \@setminipage}}%
 {\par\unskip\endMakeFramed%
 \at@end@of@kframe}
\makeatother

\definecolor{shadecolor}{rgb}{.97, .97, .97}
\definecolor{messagecolor}{rgb}{0, 0, 0}
\definecolor{warningcolor}{rgb}{1, 0, 1}
\definecolor{errorcolor}{rgb}{1, 0, 0}
\newenvironment{knitrout}{}{} % an empty environment to be redefined in TeX

\usepackage{alltt}
\usepackage{hyperref}

\title{Some title}
\author{My name}
\IfFileExists{upquote.sty}{\usepackage{upquote}}{}
\begin{document}
%\SweaveOpts{concordance=TRUE}

\maketitle

\section{Introduction}

Each team will determine what resources are available and/or needed to address the driving question. Below I have outlined the process to develop our project which begins with indvidual research, then teamwork to partition effort, and finally an ``individualized project plan.'' The plan will be worth 15\% of the project grade. I will evaluate is based on clarity, completeness, and likelihood of success. 

This plan will have the following sections: 

\begin{description}
  \item[Question to be Addressed] A clearly articulated question that we can address given our resources. I recommend that you frame it as a question AND a testable hypothesis. A testable null hypothesis is simple a thoughtful expectation that can easily proven incorrect. For example, we might want to answer the following question -- Do Santa Ana sucker each diatoms?  To revise the question, we articulate the null hypothesis as ``Santa Ana sucker have recognizable diatom cells in their gut.'' Notice that this hypothesis has a specific outcome and a good indication of the method. Of course, it doesn't really answer the question, but it might be as close as we can get. Coming up with a good question and null hypothesis comes with experience, i.e. it's a learned skill. So, you might find the first few you come up with don't do what you want.   
  \item[Approach] A narrative that describe a reasonable approach to test they hypothesis. This might discuss how many times we go in the field, where we go and how many people are needed. What types of equipment or method development you might need and how these will be used to test the null hypothesis. Of course, it will also require some acknowledgement that we have the resources to conduct the work. 
  \item[Cited Methods] Include an annoated bibliography that references several papers, describing how the methods and equipment were used to test their hypothesis or answer their question. Be sure to highlight how their question might or might not align with what you have in mind.
\end{description}
\section{Background}

\section{Question to be Addressed}

\section{Hypothesis/Hypotheses}

\section{Approach}

 

\section{Methods w/Citations}

\paragraph{Equipment}

\paragraph{Sampling}


\section{Expected Results and Analysis Methods}


\end{document}
