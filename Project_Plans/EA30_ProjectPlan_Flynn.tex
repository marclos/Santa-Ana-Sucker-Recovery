\documentclass{tufte-handout}
\usepackage{hyperref}

\title{Vieography Project Proposal}
\author{Clare Flynn}

\usepackage{Sweave}
\begin{document}
\Sconcordance{concordance:EA30_ProjectPlan_Flynn.tex:EA30_ProjectPlan_Flynn.Rnw:%
1 6 1 1 0 45 1}

%\SweaveOpts{concordance=TRUE}

\maketitle

\section{Introduction}

Each team will determine what resources are available and/or needed to address the driving question. Below I have outlined the process to develop our project which begins with indvidual research, then teamwork to partition effort, and finally an ``individualized project plan.'' The plan will be worth 15\% of the project grade. I will evaluate is based on clarity, completeness, and likelihood of success. 

This plan will have the following sections: 

\begin{description}
  \item[Question to be Addressed] A clearly articulated question that we can address given our resources. I recommend that you frame it as a question AND a testable hypothesis. A testable null hypothesis is simple a thoughtful expectation that can easily proven incorrect. For example, we might want to answer the following question -- Do Santa Ana sucker each diatoms?  To revise the question, we articulate the null hypothesis as ``Santa Ana sucker have recognizable diatom cells in their gut.'' Notice that this hypothesis has a specific outcome and a good indication of the method. Of course, it doesn't really answer the question, but it might be as close as we can get. Coming up with a good question and null hypothesis comes with experience, i.e. it's a learned skill. So, you might find the first few you come up with don't do what you want.   
  \item[Approach] A narrative that describe a reasonable approach to test they hypothesis. This might discuss how many times we go in the field, where we go and how many people are needed. What types of equipment or method development you might need and how these will be used to test the null hypothesis. Of course, it will also require some acknowledgement that we have the resources to conduct the work. 
  \item[Cited Methods] Include an annoated bibliography that references several papers, describing how the methods and equipment were used to test their hypothesis or answer their question. Be sure to highlight how their question might or might not align with what you have in mind.
\end{description}
\section{Background}  In the 1960s, the habitat of the suckers would range from 10-26°C, but when we tested the water in September 2016, we found temperatures in the Santa Ana river up to 35.5°C (Greenfield & Ross & Deckert 1970).  This rise in temperature is most likely due to the extreme industrialization of the stream; much of it runs over concrete, which heats up to extreme temperatures in the sunlight.  The river is also greatly diminished from what it used to be due to the extreme drought in Southern California.  The river is shallower, slower moving, and has less ice melt coming from the mountains, all of which factor in to an increase in temperature.  Our goal is to discover whether or not the Santa Ana suckers are coping with this dramatic increase in temperature by moving to cooler sections of the stream throughout the day.  

\section{Question} What temperature does the Santa Ana sucker prefer, and will they move throughout the day to reach their preferred temperature?

\section{Hypothesis} We expect to see the most fish wherever the water is coolest.  We also expect that during the day, the fish will move towrds the cooler temperatures.  Later in the day, when the stream is hotter, we expect to see more fish further downstream, seeking relief from the heat, than we saw earlier in the day when the water was cooler.  In order to analyze the distribution of fish in the river, we will test the null hypothesis that the number of fish present has no correlation with location in the stream or temperature of the water.

\section{Approach} We will place two gopros in two different locations in the stream to monitor the population densities of the suckers. We will record for three different two hour segments throughout the day, one in the early morning, one late morning, and one late afernoon.  Someone will go into the field in the late morning, set up the cameras, and press record.  Someone else will go back in the afternoon, switch out the memory cards and battery packs, and press record. Overnight, we will clear the memory card and recharge the battery packs.  One person will then go back early the next morning, switch out the memory cards and the battery packs, and press record.  Someone will collect the cameras within the next few days.  We will mount the cameras inside cinderblocks using gorilla glue to hide the cameras. We will need the data of our fellow classmates on water temperature during these periods to compare with our findings of the density of fish throughout the day.

 

\section{Methods w/Citations}

\paragraph{Equiptment}  \begin{itemize}
\item 2 Waterproof GoPro Hero 4 Silver cameras with mounts
\item 4 64GB memory cards
\item 4 Waterproof 6-hour battery packs compatible with GoPro
\item 2 HOBO Tidbit water temperature data loggers
\item 2 Cinderblocks
\item 2 Cinderblock backs
\item 1 bottle of Gorilla Glue
\end{itemize}

\paragraph{Sampling}


\section{Expected Results and Analysis Methods}


\end{document}
