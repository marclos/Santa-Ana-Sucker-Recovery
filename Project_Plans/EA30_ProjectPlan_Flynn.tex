\documentclass{tufte-handout}
\usepackage{hyperref}

\title{Vieography Project Proposal}
\author{Clare Flynn}

\usepackage{Sweave}
\begin{document}
\Sconcordance{concordance:EA30_ProjectPlan_Flynn.tex:EA30_ProjectPlan_Flynn.Rnw:%
1 6 1 1 0 45 1}



\maketitle

\section{Background}  The Santa Ana sucker is a 16 cm fish that lives in the rivers of Southern California.  They have recently been placed on the endangered species list, partially due to their losing over 70 percent of their habitat (FWS 2012). Also, in the 1960s, the habitat of the suckers would range from 10-26°C, but when we tested the water in September 2016, we found temperatures in the Santa Ana river up to 35.5°C (Greenfield & Ross & Deckert 1970).  This rise in temperature is most likely due to the extreme industrialization of the stream; much of it runs over concrete, which heats up to extreme temperatures in the sunlight.  The river is also greatly diminished from what it used to be due to the extreme drought in Southern California.  The river is shallower, slower moving, and has less ice melt coming from the mountains, all of which factor in to an increase in temperature.  Our goal is to discover whether or not the Santa Ana suckers are coping with this dramatic increase in temperature by moving to cooler sections of the stream throughout the day.  
Greenfield, D. W., Ross, S. T., & Deckert, G. D. (1970). Some aspects of the life history of the Santa Ana sucker, Catostomus (Pantosteus) santaanae (Snyder). Calif. Fish Game, 56(3), 166-179.

\section{Question} What temperature does the Santa Ana sucker prefer, and will they move throughout the day to reach their preferred temperature?

\section{Hypothesis} We expect to see the most fish wherever the water is coolest.  We also expect that during the day, the fish will move towrds the cooler temperatures.  Later in the day, when the stream is hotter, we expect to see more fish further downstream, seeking relief from the heat, than we saw earlier in the day when the water was cooler.  In order to analyze the distribution of fish in the river, we will test the null hypothesis that the number of fish present has no correlation with location in the stream or temperature of the water.

\section{Approach} We will place two gopros in two different locations in the stream to monitor the population densities of the suckers. We will record for three different two hour segments throughout the day, one in the early morning, one late morning, and one late afernoon.  Someone will go into the field in the late morning, set up the cameras, and press record.  Someone else will go back in the afternoon, switch out the memory cards and battery packs, and press record. Overnight, we will clear the memory card and recharge the battery packs.  One person will then go back early the next morning, switch out the memory cards and the battery packs, and press record.  Someone will collect the cameras within the next few days.  We will mount the cameras inside cinderblocks using gorilla glue to hide the cameras. We will need the data of our fellow classmates on water temperature during these periods to compare with our findings of the density of fish throughout the day.

 

\section{Methods w/Citations} Cooke, S. J., & Schreer, J. F. (2002). Determination of fish community composition in the untempered regions of a thermal effluent canal–The efficacy of a fixed underwater videography system. Environmental Monitoring and Assessment, 73(2), 109-129.  DOI: 10.1023/A:1013099430900
Their goal was to monitor the number and behavior of many different species of fish in thermal discharge canals.  They reviewed many different techniques for monitoring fish, and came to the conclusion that fixed underwater videography was the most effective.  They also suggested a wide angle lens in order to capture the most data, which we plan to incorporate.  However, they were analyzing behavior, so they only watched 10 min segments of videos at slow speed to note behavior.  We will be skipping through frames of a much longer segment since we only need to count the fish. 

Yoshida, T., Akagi, K., Toda, T., Kushairi, M. M. R., Kee, A. A. A., & Othman, B. H. R. (2010). Evaluation of fish behaviour and aggregation by underwater videography in an artificial reef in Tioman island, Malaysia. Sains Malaysiana, 39(3), 395-403.
They were recording the fish behavior around an artificial reef.  They said 160 cm was a good distance in terms of visibility, but our river will probably have different levels of sedimentation than their ocean, so we should play around with the distance.  They also suggested shooting at a diagonal angle saying it increases visibility.  We can also experiment with the angle, but the river is most likely too shallow to use any angle but straight.

Bergman, P. (2011). Videography monitoring of adult sturgeon in the Feather River basin, CA. Report to Anadromous Fish Restoration Program, Cramer Fish Sciences, Gresham, Oregon.
Their goal was to observe the presence of sturgeon in areas with different bottom substrate.  They attached the camera to their boat, then recorded as they moved downstream.  They pointed out that they wouldn’t be able to properly estimate the abundance of fish since they created a disturbance with the boat.  This made us realize we cannot look at the footage directly before or after we walk through the stream.

Chamness, M., Abernethy, S., & Tunnicliffe, C. (2007). Walla Walla River Basin Fish Screens Evaluations, 2006 Annual Report (No. DOE/BP-00000652-36). Bonneville Power Administration (BPA), Portland, OR (United States).  DOI:10.2172/961999
This group examined the presence of fish in the river by walking up the river with a camera mounted on a pole.  While this technique would have provided us with many more locations to analyze, it would have disturbed the natural habitat of the fish, and possibly scared them off.  It also would have taken a lot more time.



\paragraph{Equiptment}  \begin{itemize}
\item 2 Waterproof GoPro Hero 4 Silver cameras with mounts
\item 4 64GB memory cards
\item 4 Waterproof 6-hour battery packs compatible with GoPro
\item 2 HOBO Tidbit water temperature data loggers
\item 2 Cinderblocks
\item 2 Cinderblock backs
\item 1 bottle of Gorilla Glue
\end{itemize}

\paragraph{Sampling} We will record at two different locations for three different time frames.  We will examine footage from two hours of each time frame, 8-10 am, 11-1 pm, and 2:30-4:30 pm.  EVery minute, we will pause the footage and count the number of fish in the screen.  If we are unable to tell from the stillshot, we will watch the video leading up to it to look for movement.  We will then take the average of the number of fish from each counted frame.  We should have six different averages, early upstream, early downstream, middle of the day upstream, middle downstream, afternoon upstream, and afternoon downstream.  


\section{Expected Results and Analysis Methods} We hope to be able to reject the null hypothesis that time and location have no effect on quantity of fish.  This would mean that there is a correlation between location, time, and number of fish.  Our independent variable is the 6 different time and locations we sampled, which is categorical, and our dependent variable is the number of fish, which is continuous.  This means we will preform an ANOVA test to analyze the data.  We expect to see significantly more fish downstream later in the day than earlier, and less fish upstream later in the day than earlier.


\end{document}
